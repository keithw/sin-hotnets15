\section{Related work}
\label{sec:related}

Our work follows a long line of research in network economics,
interdomain routing, and congestion pricing.  We summarize some of the
most closely related contributions:

\subsection{Paid interdomain routing}

Route Bazaar~\cite{routebazaar15} uses a Bitcoin-like public ledger to
allow ASes to advertize pathlets and SLA's automatically to negotiate
and verify connectivity agreements for fixed volumes of traffic. This
work follows previous work in Pathlet Routing~\cite{pathlet09}. We
view our work as complementary to these efforts. We focus on resource
allocation within the last mile or a small number of links---problems
that today are within the domain of congestion control and AQM, not
routing protocols.

\subsection{Incentivized mesh networks}

Kadupul~\cite{kadupul15} uses time-locked puzzles to incentivize
forwarding in a mesh network.
 without measuring latency.  identified incentives
missing to support alternate routing today. desired pricing based on
costs of transmission looks at various reward schemes for forwarding
based on time locked puzzles

\subsection{Dynamic congestion pricing}



There has also been work to incentivize participation in the Tor network\cite{torpath14, onions14} (TODO their relwork, toroken?)

%\section{Congestion pricing}


\subsection{Internet Pricing}
\subsubsection{Tiered Service}
Work in internet pricing in the 1990's focused on providing different paid tiers of service for differeng levels of Quality of Service.

In Cocchi et al's Pricing in Computer Networks: Motivation, Formulation, and Example \cite{cocchi93}, applications defined value functions for networks performance and chose between a high priority tier and a cheaper lower priority tier ( low priority packets were dropped by droptail fifo switches first ). They a two tiered pricing scheme had higher aggregate user satisfaction than a single tiered one, especially at higher network utilization.

\subsubsection{Dyanmic Pricing}
Having time, location, or congestion dependent pricing has also been studied since the 1990's.


Work in the last few years by Mung Chiang has included a survey of work on dynamic pricing of mobile data \cite{pricingdata13} and TUBE: Time-Dependent Pricing for Mobile Data \cite{tube12}.
TUBE dynamically generates a 24 hour time dependent pricing scheme for each base station for mobile data. Giving discounts to users during periods of predicted low use.
Their implementation on the user side involved a UI, data usage tracker, and an 'autopilot mode' which scheduled applications to keep them below a user specified monthly budget. It also involved software at the ISP side to measure traffic and set price schedules.
User studies showed users would usually traffic when notified by text message they were using during a high price period, but overall usage increased because users would use more mobile data during discounted periods.
Their work revolves around a system between a user and their ISP, while we desire to expose users to any available route of connectivity, encouraging competition with traditional ISPs.

\subsection{Mesh Networks}
There has also been a few attempts in using economic incentives or auctions to increase network connectivity in mesh networks.
auction for packet forwarding in mesh network, aka ad hoc netowrk \cite{anderegg03, chen04, chen05, wang06, demir07,zhong07, kargar08, zhu08, eidenbenz08, wu10, zhong10, martignon11, martignon15} aka delay tolerant network \cite{chen13}



\subsection{Interdomain Routing}
Our work is complementary to recently proposed new mechanisms for more dynamic interdomain routing policies



\subsection{Network Scheduling}
Calendaring for Wide Area Networks \cite{tempus14}
is a datacenter Traffic Engineering scheme that uses mixed packing covering solvers over a sliding window of future timesteps to meet the deadlines of long lived transfers while serving high priority traffic.
The inputs to their system include a time when the system becomes aware of the request, a time when the request can start transferring, a demand, and a deadline that can be a value or function.
In their evaluation, they "interviewed cluster operators for typical distributions of the sizes of long lived requests and the durations between their arrival and hard deadline" and to create a distribution to sample from.

\subsubsection{Scheduling Auctions}
Wellman et. al. explore using simultaneous ascending auctions to schedule a fixed number of time slots and users that have jobs with fixed lengths and a non-increasing function from time to completion to value \cite{wellman01, wellman05}.
Simultaneous auctions have bad theoretical properties because of the well-known \emph{exposure problem}, where the outcome of one auction affects the value of an item in another (TODO: CITE?).


To sort:
shenker Fundamental design issues for the future Internet \cite{shenker95}
internet pricing surverys\cite{stiller01, falkner00, odlyzko01}

Pricing Communication Networks: Economics, Technology and Modeling book \cite{courcoubetis03}

users compete for bandwidth at a link by submitting several couples (e.g., amount of bandwidth asked, associated unit price) so that the link allocates the bandwidth and computes the charge according to the second price principle \cite{maille06}


INDEX pricing experiments \cite{altmann99, edell99}

paris metro pricing \cite{odlyzko99}

smart market, a per-packet auction to solve congestion problems

pricing for qos netowrks \cite{dasilva00, marbach04}
in qos congestion dependent (qos and congestion dependent similar) \cite{shu03}
congestion dependent pricing \cite{mason95, peha97, paschalidis00, la00, la02}
Progressive Second Price auction, which breaks up bandwidth by bidders  \cite{lazar98, lazar99, semret99, semret00,maille03, bitsaki05, beltran07}

Market-driven Bandwidth Allocation in Selfish Overlay Networks \cite{wang05}
Cumulus pricing scheme \cite{reichl01, stiller01cumulus, reichl01edgepricing, reichl03, hayel05}

newer action mechanism \cite{dramitinos07}
Pricing Across Multiple ISP Domains \cite{saberi07}
WiFi Access Point Pricing as a Dynamic Game \cite{musacchio06}
old routebazaar, pricemyroute \cite{esquivel09, pricemyroute11}
market for isp peering \cite{hau09}

Incentive compatibility and dynamics of congestion control (godfrey, shenker) \cite{godfrey10}
Truthful prioritization for dynamic bandwidth sharing (for 4g) \cite{shnayder14}
there is work not cited on economic models for p2p

choicenet \cite{wolf14}
