\section{Related work}

\label{sec:related}

\subsection{Routing}

Kadupul \cite{kadupul15}

PriceMyRoute \cite{pricemyroute11}

Route Bazaar \cite{routebazaar15}

Pathlet Routing \cite{pathlet09}

\subsection{Pricing}

Work in the last few years by Mung Chiang has included a survey of work on dynamic pricing of mobile data \cite{pricingdata13} and TUBE: Time-Dependent Pricing for Mobile Data \cite{tube12}.
TUBE dynamically generates a 24 hour time dependent pricing scheme for each base station for mobile data. Giving discounts to users during periods of predicted low use.
Their implementation on the user side involved a UI, data usage tracker, and an 'autopilot mode' which scheduled applications to keep them below a user specified monthly budget. It also involved software at the ISP side to measure traffic and set price schedules.
User studies showed users would usually traffic when notified by text message they were using during a high price period, but overall usage increased because users would use more mobile data during discounted periods.
Day ahead pricing can be inaccurate (ie super bowl)

\subsection{Scheduling}

Calendaring for Wide Area Networks \cite{tempus14}
is a datacenter Traffic Engineering scheme that uses mixed packing covering solvers over a sliding window of future timesteps to meet the deadlines of long lived transfers while serving high priority traffic.
The inputs to their system include a time when the system becomes aware of the request, a time when the request can start transferring, a demand, and a deadline that can be a value or function.
In their evaluation, they "interviewed cluster operators for typical distributions of the sizes of long lived requests and the durations between their arrival and hard deadline" and to create a distribution to sample from.

\subsection{Auctions}
Auction Protocols for Decentralized Scheduling \cite{wellman01}
