\section{Related work}
\label{sec:related}

Our work follows a long line of research in network economics,
interdomain routing, and congestion pricing.  We summarize some of the
most closely related contributions:

\subsection{Incentivized mesh networks}
Much work has been done since the 2000s to explore incentivizing selfish nodes to participate in forwarding in a mobile ad-hoc network.
SPRITE~\cite{zhong03} used a credit system and determined prices from a game-theoretic perspective. They showed their system motivated honest participation, even when a collection of the selfish nodes collude. Ad hoc-VCG used an auction mechanism
and found that this achieved higher efficiency~\cite{anderegg03}. More recently, Kadupul~\cite{kadupul15} explored the use time-locked puzzles to incentivize fast forwarding in a mesh network,
and others have developed schemes to encourage relaying in the Tor network while discouraging cheaters who
advertise phony capacity~\cite{torpath14, onions14}.
%A large body of work subsequently investigated problems of collusion, security, path discovery, and verification~\cite{buttyan03,chen04, chen05, wang06, demir07,zhong07, kargar08, zhu08, eidenbenz08, wu10, zhong10, martignon11, martignon15, chen13}. Our work adds to this area by investigating the ability of applications and endpoints to express their desires and bid and offer packet-transmission opportunities accordingly.

\subsection{Paid interdomain routing}
%A BGP-based Mechanism for Lowest-Cost Routing \cite{feigenbaum05} assumed a cost for each packet transit between \emph{Autonimous Systems} that was computed using an extension to BGP with minimal communication overhead. They showed their pricing mechanism elicited a truthful revelation of AS transit costs and that they could efficiently compute the lowest cost paths for all source-destination pairs.

Route Bazaar~\cite{routebazaar15} uses a Bitcoin-like public ledger to
allow autonomous systems to advertise pathlets and SLAs
automatically, in order to negotiate and verify connectivity
agreements for traffic aggregates. This work follows previous work in
Pathlet Routing~\cite{pathlet09}. We view this work as complementary;
in this paper, we focus on resource-allocation problems typically
solved by TCP and AQM over a single hop or last-mile network,
deferring discussion of routing or multi-hop situations.

\subsection{Tiered service and dynamic pricing}
Work in Internet pricing in the 1990s focused on providing different
paid tiers of service for differing levels of Quality of Service.  We
were inspired by~\cite{cocchi93}, where applications defined value
functions that endpoints used to choose between high and low priority
priced tiers of internet service, leading to higher aggregate user satisfaction.

More recent work on pricing has focused on cellular
Internet. TUBE~\cite{tube12} dynamically generated a 24-hour
time-dependent pricing scheme for each base station, giving discounts
to users during periods of predicted low traffic. User studies showed that users would
usually reduce consumption when notified by text message of high prices.
A user interface like the
one in TUBE would be a valuable to a real world implementation of
self-incentivizing networks. Our Bidding-Control Protocols are in
essence a more fine-grained version of their 'autopilot mode' for scheduling
applications.  Our work expands on these notions by adding
the possibility that anybody may act as ISP, perhaps temporarily, and convey an application's traffic.

\subsection{Objective- and auction-based scheduling}
Tempus~\cite{tempus14} is a datacenter traffic-engineering scheme built to simultaneously meet the deadlines of long-lived transfers while serving high-priority traffic by using an offline scheduler.
The authors showed, in a non-adversarial setting, that smarter scheduling can produce better overall utility for users with diverse needs. We hope to achieve similar goals but in a greedy, adversarial setting.

Structurally similar to our packet-by-packet market in the next section,
Wellman et.~al.~explore using simultaneous ascending auctions to
schedule a fixed number of time slots and users that have jobs with
fixed lengths and a non-increasing function from time to completion to
value \cite{wellman01, wellman05}. Simultaneous auctions have
unfortunate theoretical properties because of the well-known
\emph{exposure problem}, where the outcome of one auction affects the
value of an item in another~\cite{milgrom00, englmaier06}. We discuss how this problem affects
our own market in Section~\ref{ss:eval}.






















\begin{comment}
%\subsection{Paid tiered service}

%\section{Congestion pricing}


%\subsection{Internet Pricing}
%\subsubsection{Tiered Service}

\subsubsection{Dynamic Pricing}
Having time, location, or congestion dependent pricing has also been studied since the 1990's.


Work in the last few years by Mung Chiang has included a survey of work on dynamic pricing of mobile data \cite{pricingdata13} and TUBE: Time-Dependent Pricing for Mobile Data \cite{tube12}.
TUBE dynamically generates a 24 hour time dependent pricing scheme for each base station for mobile data. Giving discounts to users during periods of predicted low use.
Their implementation on the user side involved a UI, data usage tracker, and an 'autopilot mode' which scheduled applications to keep them below a user specified monthly budget. It also involved software at the ISP side to measure traffic and set price schedules.
User studies showed users would usually traffic when notified by text message they were using during a high price period, but overall usage increased because users would use more mobile data during discounted periods.
Their work revolves around a system between a user and their ISP, while we desire to expose users to any available route of connectivity, encouraging competition with traditional ISPs.

\subsection{Mesh Networks}
There has also been a few attempts in using economic incentives or auctions to increase network connectivity in mesh networks.
auction for packet forwarding in mesh network, aka ad hoc netowrk \cite{anderegg03, chen04, chen05, wang06, demir07,zhong07, kargar08, zhu08, eidenbenz08, wu10, zhong10, martignon11, martignon15} aka delay tolerant network \cite{chen13}



\subsection{Interdomain Routing}
Our work is complementary to recently proposed new mechanisms for more dynamic interdomain routing policies



\subsection{Network Scheduling}
\subsubsection{Scheduling Auctions}
Simultaneous auctions have bad theoretical properties because of the well-known \emph{exposure problem}, where the outcome of one auction affects the value of an item in another (TODO: CITE?).


To sort:
shenker Fundamental design issues for the future Internet \cite{shenker95}
internet pricing surverys\cite{stiller01, falkner00, odlyzko01}

Pricing Communication Networks: Economics, Technology and Modeling book \cite{courcoubetis03}

users compete for bandwidth at a link by submitting several couples (e.g., amount of bandwidth asked, associated unit price) so that the link allocates the bandwidth and computes the charge according to the second price principle \cite{maille06}


INDEX pricing experiments \cite{altmann99, edell99}

paris metro pricing \cite{odlyzko99}

smart market, a per-packet auction to solve congestion problems

pricing for qos netowrks \cite{dasilva00, marbach04}
in qos congestion dependent (qos and congestion dependent similar) \cite{shu03}
congestion dependent pricing \cite{mason95, peha97, paschalidis00, la00, la02}
Progressive Second Price auction, which breaks up bandwidth by bidders  \cite{lazar98, lazar99, semret99, semret00,maille03, bitsaki05, beltran07}

Market-driven Bandwidth Allocation in Selfish Overlay Networks \cite{wang05}
Cumulus pricing scheme \cite{reichl01, stiller01cumulus, reichl01edgepricing, reichl03, hayel05}

newer action mechanism \cite{dramitinos07}
Pricing Across Multiple ISP Domains \cite{saberi07}
WiFi Access Point Pricing as a Dynamic Game \cite{musacchio06}
old routebazaar, pricemyroute \cite{esquivel09, pricemyroute11}
market for isp peering \cite{hau09}

Incentive compatibility and dynamics of congestion control (godfrey, shenker) \cite{godfrey10}
Truthful prioritization for dynamic bandwidth sharing (for 4g) \cite{shnayder14}
there is work not cited on economic models for p2p

choicenet \cite{wolf14}
\end{comment}
