\section{Related work}
\label{sec:related}

Our work follows a long line of research in network economics,
interdomain routing, and congestion pricing.  We summarize some of the
most closely related contributions:

\subsection{Incentivized mesh networks}
Much work has been done since the 2000's to explore incentivizing selfish nodes to participate in forwarding in a mobile ad-hoc network.
SPRITE~\cite{zhong03} used a credit system and determined prices from a game-theoretic perspective. They showed thier system motivated honest participation, even when a collection of the selfish nodes collude. Ad hoc-VCG used Vickrey, Clarke, and Groves for packet trasmission and claimed this achieved higher economic efficiency~\cite{anderegg03}.
A large body of work followed these early findings that is complementary to our work~\cite{buttyan03,chen04, chen05, wang06, demir07,zhong07, kargar08, zhu08, eidenbenz08, wu10, zhong10, martignon11, martignon15, chen13} and has looked into problems of collusion, security, incentivized path discovery, and verification. What we bring that the incentivized ad-hoc networking doesn't is the ability for applications to define their goals from the network and bid accordingly.

More recently, Kadupul~\cite{kadupul15} shared many of our views on incentives and explored the use time-locked puzzles to incentivize fast forwarding in a mesh network without measuring latency.
There has also been work by Ford et.~al. in incentivizing participation in the Tor network\cite{torpath14, onions14}.

\subsection{Paid interdomain routing}
A BGP-based Mechanism for Lowest-Cost Routing \cite{feigenbaum05} assumed a cost for each packet transit between \emph{Autonimous Systems} that was computed using an extension to BGP with minimal communication overhead. They showed their pricing mechanism elicited a truthful revelation of AS transit costs and that they could efficiently compute the lowest cost paths for all source-destination pairs.

Route Bazaar~\cite{routebazaar15} uses a Bitcoin-like public ledger to
allow ASes to advertize pathlets and SLA's automatically to negotiate
and verify connectivity agreements for fixed volumes of traffic. This
work follows previous work in Pathlet Routing~\cite{pathlet09}, which introduced the primative of fragments of paths between virtual nodes that can be used to simplify existing protocols like BGP or to facilitate the addition of new features like multipath and source routing.
We view paid interdomain routing work as complementary to ours. Our vision is to use market mechanisms for all routing across the internet.
Our current work focused on resource allocation in the last mile and for a small number of links---problems that today are within the domain of congestion control and AQM, not routing protocols.
Future work in wide area routing would expand on existing work paid and automatic interdomain routing.

\subsection{Tiered service and dynamic pricing}
Work in internet pricing in the 1990's focused on providing different paid tiers of service for differing levels of Quality of Service.
We were inspired by how applications defined value functions in Pricing in Computer Networks: Motivation, Formulation, and Example \cite{cocchi93}. In that paper, they used them to chose between a high priority tier and a cheaper lower priority tier, showing the two tiered pricing scheme had higher aggregate user satisfaction than the single tiered one, especially at higher network utilization.

More recent work on pricing has focused on cellular internet. TUBE \cite{tube12} dynamically generated a 24 hour time dependent pricing scheme for each base station, giving discounts to users during periods of predicted low use.
Their implementation on the user side involved a UI, data usage tracker, and an 'autopilot mode' which scheduled applications to keep them below a user specified monthly budget. It also involved software at the ISP side to measure traffic and set price schedules.
User studies showed users would usually traffic when notified by text message they were using during a high price period, but overall usage increased because users would use more mobile data during discounted periods.
A user interface like the one in TUBE would be a valuable to a real world implementation of self-incentivizing networks. Our Bidding Control Protocols are in essence a more fine grained version of an 'autopilot mode.'
However, both these works revolve around a system between a user and their ISP, while we desire to expose users to any available route of connectivity, encouraging competition with traditional ISPs that we believe will reduce costs for users and increase the overall efficiency of the network.
\subsection{Objective- and auction-scheduling}
Tempus \cite{tempus14} is a datacenter Traffic Engineering scheme that uses mixed packing covering solvers over a sliding window of future timesteps to meet the deadlines of long lived transfers while serving high priority traffic.
The inputs to their system include a time when the system becomes aware of the request, a time when the request can start transferring, a demand, and a deadline that can be a value or function.
In their evaluation, they "interviewed cluster operators for typical
distributions of the sizes of long lived requests and the durations
between their arrival and hard deadline" and to create a distribution
to sample from.

Wellman et. al. explore using simultaneous ascending auctions to schedule a fixed number of time slots and users that have jobs with fixed lengths and a non-increasing function from time to completion to value \cite{wellman01, wellman05}.
Simultaneous auctions have bad theoretical properties because of the well-known \emph{exposure problem}, where the outcome of one auction affects the value of an item in another (TODO: CITE?).


\begin{comment}
%\subsection{Paid tiered service}

%\section{Congestion pricing}


%\subsection{Internet Pricing}
%\subsubsection{Tiered Service}

\subsubsection{Dyanmic Pricing}
Having time, location, or congestion dependent pricing has also been studied since the 1990's.


Work in the last few years by Mung Chiang has included a survey of work on dynamic pricing of mobile data \cite{pricingdata13} and TUBE: Time-Dependent Pricing for Mobile Data \cite{tube12}.
TUBE dynamically generates a 24 hour time dependent pricing scheme for each base station for mobile data. Giving discounts to users during periods of predicted low use.
Their implementation on the user side involved a UI, data usage tracker, and an 'autopilot mode' which scheduled applications to keep them below a user specified monthly budget. It also involved software at the ISP side to measure traffic and set price schedules.
User studies showed users would usually traffic when notified by text message they were using during a high price period, but overall usage increased because users would use more mobile data during discounted periods.
Their work revolves around a system between a user and their ISP, while we desire to expose users to any available route of connectivity, encouraging competition with traditional ISPs.

\subsection{Mesh Networks}
There has also been a few attempts in using economic incentives or auctions to increase network connectivity in mesh networks.
auction for packet forwarding in mesh network, aka ad hoc netowrk \cite{anderegg03, chen04, chen05, wang06, demir07,zhong07, kargar08, zhu08, eidenbenz08, wu10, zhong10, martignon11, martignon15} aka delay tolerant network \cite{chen13}



\subsection{Interdomain Routing}
Our work is complementary to recently proposed new mechanisms for more dynamic interdomain routing policies



\subsection{Network Scheduling}
\subsubsection{Scheduling Auctions}
Simultaneous auctions have bad theoretical properties because of the well-known \emph{exposure problem}, where the outcome of one auction affects the value of an item in another (TODO: CITE?).


To sort:
shenker Fundamental design issues for the future Internet \cite{shenker95}
internet pricing surverys\cite{stiller01, falkner00, odlyzko01}

Pricing Communication Networks: Economics, Technology and Modeling book \cite{courcoubetis03}

users compete for bandwidth at a link by submitting several couples (e.g., amount of bandwidth asked, associated unit price) so that the link allocates the bandwidth and computes the charge according to the second price principle \cite{maille06}


INDEX pricing experiments \cite{altmann99, edell99}

paris metro pricing \cite{odlyzko99}

smart market, a per-packet auction to solve congestion problems

pricing for qos netowrks \cite{dasilva00, marbach04}
in qos congestion dependent (qos and congestion dependent similar) \cite{shu03}
congestion dependent pricing \cite{mason95, peha97, paschalidis00, la00, la02}
Progressive Second Price auction, which breaks up bandwidth by bidders  \cite{lazar98, lazar99, semret99, semret00,maille03, bitsaki05, beltran07}

Market-driven Bandwidth Allocation in Selfish Overlay Networks \cite{wang05}
Cumulus pricing scheme \cite{reichl01, stiller01cumulus, reichl01edgepricing, reichl03, hayel05}

newer action mechanism \cite{dramitinos07}
Pricing Across Multiple ISP Domains \cite{saberi07}
WiFi Access Point Pricing as a Dynamic Game \cite{musacchio06}
old routebazaar, pricemyroute \cite{esquivel09, pricemyroute11}
market for isp peering \cite{hau09}

Incentive compatibility and dynamics of congestion control (godfrey, shenker) \cite{godfrey10}
Truthful prioritization for dynamic bandwidth sharing (for 4g) \cite{shnayder14}
there is work not cited on economic models for p2p

choicenet \cite{wolf14}
\end{comment}
