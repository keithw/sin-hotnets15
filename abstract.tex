\subsection*{Abstract}
We argue that the networking research community should pursue "self-incentivizing networks," where anybody may contribute incremental capacity between pairs of endpoints and be rewarded for their contribution through a market matching engine.
Applications will be able to express their desires and bid for differentiated or guaranteed services, but will be agnostic as to who provides the service -- creating a market incentive to build out capacity where the network needs it most.
Contrary to some interpretations of network neutrality, our view is that "best-effort delivery" should be replaced by "paid prioritization" as the *only* service advertised by the network.

For the matching engine, we propose a multiresolution packet-by-packet market for carriage, in which anybody may "offer" the right to transmit a packet from point A to point B, and applications bid on the time granularity that expresses their preferred tradeoff between immediacy and cost.
In simulation experiments, this market achieved near-optimal flow-completion times and significantly outperformed flow-rate fair systems.

