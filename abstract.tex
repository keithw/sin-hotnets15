\subsection*{Abstract}
We argue that computer networks should be made ``self-incentivizing'':
designed so that anybody may contribute incremental capacity and be
rewarded for it.

In such a scheme, endpoints express their demand in terms of a price,
not just for ``bandwidth,'' but effectively by bidding on each packet
they want transmitted. This creates a market incentive to build out
capacity where the network needs it most.

Our position is that this kind of mechanism---one where ``paid
priorization'' is celebrated instead of banned, and anybody may be an
ISP-for-a-day---is the missing piece necessary to revive
almost-successful systems of the recent past, such as urban mesh
networks.

We report the results of simulation experiments where users bid for
transmission opportunities in a logically-centralized order book. In a
model where users behave ``nicely,'' this distributed-bidding scheme
was sufficient to realize schedules that closely approximated the
shortest-remaining-time-first schedule.

We then describe a multiresolution market design that aims to function
in the presence of strategic bidders and users with differing utility functions.
