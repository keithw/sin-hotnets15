\section{Limitations and future work}
\label{sec:limits}

This position paper is a combination of several things: (1) a policy
argument that network neutrality ought to include neutrality towards
\textbf{customer} packets and a ``right to resell,'' (2) an
architectural sketch about how that right could lead to a
Self-Incentivizing Network and ameliorate some vexing problems, such
as the difficulty in deploying guaranteed services and mesh networks,
and (3) simulation experiments that suggest that a SIN market among
greedy endpoints can produce results at least as good as the status
quo (in fact, better than an idealized TCP), at least to the extent
those endpoints behave honestly.

Each of these pieces would fairly be described as preliminary. On the
policy angle, we have not attempted an economic analysis to explain
whether consumer Internet access would cost more if retail ISPs were
forbidden to ban reselling.

On the architectural design and analysis, we have limited our
discussion to the case of scheduling a single hop and have not
discussed routing or topology-discovery problems at all, although we
view our work as complementary to related work that tackles these
issues~\cite{routebazaar15, esquivel09}.

In future work, we intend to show rigorously that the multiresolution
market analytically satisfies a bounded incentive-compatibility
property, such that strategic bidders can only inflict a
logarithmically-bounded amount of pain or advantage for themselves. We
think the hierarchical nature of the market will make such a showing
possible, but have not simulated or proved anything about these
markets yet, beyond pencil-and-paper experiments.

We have not discussed the problem of ``cheating''---if anybody can be
an ISP-for-a-day and offer to convey packets, what stops somebody from
taking money for packets that they then drop? How can the system
punish this behavior? Others have dealt with this problem in the
context of mesh networking~\cite{onions14, torpath14, kadupul15}, and
we expect we will need to leverage similar solutions.

We expect that the first deployment of a self-incentivizing network
would be in a friendlier domain, such as the ``Neighbor
sharing'' case from Section~\ref{ss:towards}, where some of these
challenging game-theoretic issues might be deferred.
