\section{Conclusions}
The large body of work on paid QoS in the 1990s and the proposed designs of economically incentivized mesh networks in the 2000s all contained what seemed like good and necessary ideas for the future success of the Internet, yet they never caught on.
The Internet has become even more ubiquitous today, but we believe its resource efficiency and potential to add real value to it's users is being held back by the lack of routing options and ability for applications to express their needs from the network.
Cable internet service is monopolized by one provider in a large portion of the United States, the internet in developing countries is really bad (TODO somethign else), and video confrencing is still rarely a smooth experience.

We think the literature on application driven QoS and incentivized networks needs to be brought together and viewed holistically to [do great things]. We see application driven QoS and incentivized (mesh) networking as ideas that add value to each other and are most likely to succeed when implemented together.

We desire a more fluid and efficient internet, with real guaruntees of service instead of false promises. We see a future where mobile devices are constantly chosing the best routes to achieve a good user experience from all connectivity options around them.
We think any connectivity shortage should be magnet to entrepreneurs: when cell carriers have trouble supplying capacity during a parade in San Francisco or an oppresive government shuts down the main internet network nationwide, self-incentivizing neworks should spring up serendipitously to users and those users should be able to seamlessly chose routes for for their packets and evaluate them on more than just bandwidth per dollar.

%In this paper, we introduced...
%A set of flow completion time users achieve near optimal mean flow duration over a shared link while compensating users of long flows to be pre-empted

%for pricing to reduce congestion software of users needs to adapt its behavior to dynamic pricing.

%things greg wants to have covered here or elsewhere:
%applications should define their own objectives.

%The interface we see to the users is as simple as: how much to spend per day/month, and something that lets people see how much money different applications are spending. If OS enforces reasonable limits, no runaway spending, only applications that can take more than their fair share, which is no different than situation today

%Other advanced network features such as caching and multicast could be served by a competetive market
\label{sec:conc}
