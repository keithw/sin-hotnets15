\section{Conclusion}
The large body of work on paid QoS in the 1990s, and the urban mesh
networks of the 2000s, all contained good ideas for the future success
of the Internet, but did not lead to sustainably deployed systems. And
despite the meteoric growth of the Internet in the last 20 years, the
inability to receive bandwidth or latency guarantees continues to harm
real-time interactive applications.

In this position paper, we argue in essence that the movement for
``network neutrality'' got it backwards. The promising direction, we
believe, involves designing networks and telecom policies so that
anybody---even a retail ISP customer---may contribute incremental
capacity and be rewarded for it.

Our position is that this kind of self-incentivizing network would be a
missing piece helpful in reviving almost-successful systems of the
recent past, such as urban mesh networks, and fostering more innovation
and growth in Internet service.

\label{sec:conc}
