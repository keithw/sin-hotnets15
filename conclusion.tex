\section{Conclusions}
While the 90's saw a large body of work on paid QoS and economically incentivized mesh networks were studied in the 2000's, what seemed like good and necessary ideas for success never caught on. The internet has become even more ubiquitous, we believe its resource efficiency and potential to add value to the lives of it's users is being held back by the same issues brought up in the QoS and ad hoc networking literature. Cable internet service is monopolized by one provider in a large portion of the United States, the internet in developing countries is really bad (TODO somethign else), and video confrencing is still rarely a smooth experience.

We see application driven QoS and incentivized (mesh) networking as ideas that add value to each other and are most likely to succeed when implemented together.

A set of flow completion time users achieve near optimal mean flow duration over a shared link while compensating users of long flows to be pre-empted

for pricing to reduce congestion software of users needs to adapt its behavior to dynamic pricing.

things greg wants to have covered here or elsewhere:
applications should define their own objectives.

The interface we see to the users is as simple as: how much to spend per day/month, and something that lets people see how much money different applications are spending. If OS enforces reasonable limits, no runaway spending, only applications that can take more than their fair share, which is no different than situation today

hopefully phones could some day get 4g from independent people, walking around seamlessly connect to wifi networks, and you can make a skype call and have it work for once

hopefully phones could some day get 4g from independent people, walking around seamlessly connect to wifi networks, and you can make a skype call and have it work for once

A large public gathering would be a magnet to entrepreneurs. When cell carriers have trouble supplying capacity during a parade in downtown san francisco, users could find other paid routes for communication. Mobile-ad hoc networks would spring up serendipitously to everyday users when the demand existed.

Other advanced network features such as caching and multicast could be served by a competetive market
\label{sec:conc}
