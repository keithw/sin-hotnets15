\section{Introduction}

\label{sec:intro}

On June 12, 2015, the Federal Communications Commission's Open
Internet rules went into effect. In pursuit of what's commonly known
as ``net neutrality,'' the agency bans ``retail'' Internet service
providers from ``blocking,'' ``throttling,'' or ``paid
prioritization'' of traffic across their networks~\cite{openinternet}.

The ``paid priorization'' rule protects third-party
content providers who aren't transit customers or peers of the
ISP. The rule prohibits retail ISPs from demanding money from a third
party or affiliate in exchange for ``traffic shaping, prioritization,
resource reservation, or other forms of preferential traffic
management'' within the ISPs network.

An unaffiliated third party can send its own traffic, or somebody
else's traffic---the ISP cannot discriminate except to the extent that
it has a link at its border (a direct peering or transit connection to
the sender) that it can constrict wholesale.

In essence, the FCC's rule requires that ISPs behave ``neutrally''
with regards to traffic coming from content providers who
\textbf{aren't} their customers (or peers).

But what of the ISP's actual paying customers? By contrast, they
receive no guarantee of neutrality under the rule. Major ISPs freely
restrict their customers from sending packets on behalf of other
paying entities\footnote{Comcast and Google Fiber prohibit residential
  customers from reselling Internet service; Time Warner Cable does
  not.} or block certain Internet traffic to and from
customers, such as TCP port 25.

In this position paper, we argue that network neutrality got it
backwards. Internet growth is to be had by \emph{encouraging} paid
prioritization, while requiring the owners of communications link to
deal neutrally with everybody.

\subsection{Stickiness in today's Internet}

Today, Internet service is ``sticky''---a consumer typically has one
ISP and is reluctant to switch. Transaction costs to enter the market
are considerable: an ISP must negotiate peering or transit agreements
with counterparties, then work to lure customers away from
their current providers.

Although Internet access is essentially a commodity, there is
substantial evidence that consumers are swayed by noneconomic
factors. Last year, Comcast spent \$3.1 billion on advertising and
marketing for its Cable Communications business,\footnote{This figure
  does not include advertising for NBC or Universal Studios, but does
  include ads that promote both cable-TV and Internet connectivity.}
compared with \$2.0 billion in capital expenditures to improve its
cable distribution system~\cite{comcastannualreport}. By contrast,
mature commodities markets with sophisticated buyers (e.g.~coal,
natural gas) see little advertising.

At the same time, the Internet falls down at its ability to meet user
needs. Two decades ago, the IntServ RFC authors wrote that ``real-time
applications often do not work well across the Internet because of
variable queueing delays and congestion losses.''~\cite{rfc1633}
Despite meteoric growth of the Internet since then, the statement is
still true. There's no way for an ordinary Skype user to pay more to receive
guaranteed throughput and bounded latency, or for an e-mail sender to
pay less because of flexibility about when a message is sent.

In a world where an ISP provides a branded service all by itself, with
no relationship to the economic value provided by a connection,
exciting models of communications networks remain tantalizingly
unrealized.  Google's attempt at a Wi-Fi mesh network for Mountain
View, Calif., ultimately became overloaded with traffic and was shut
down in 2014~\cite{pcworld13}. Cisco Meraki's mesh network in San
Francisco suffered a similar fate~\cite{economist14}.

\subsection{Policy considerations}
