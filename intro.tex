\section{Introduction}

\label{sec:intro}

On June 12, 2015, the Federal Communications Commission's Open
Internet rules went into effect. In pursuit of what's commonly known
as ``net neutrality,'' the agency bans ``retail'' Internet service
providers from ``blocking,'' ``throttling,'' or ``paid
prioritization'' of traffic across their networks~\cite{openinternet}.

The ``paid priorization'' rule protects third-party
content providers who aren't transit customers or peers of the
ISP. The rule prohibits retail ISPs from demanding money from a third
party or affiliate in exchange for ``traffic shaping, prioritization,
resource reservation, or other forms of preferential traffic
management'' within the ISPs network.

An unaffiliated third party can send its own traffic, or somebody
else's traffic---the ISP cannot discriminate except to the extent that
it has a link at its border (a direct peering or transit connection to
the sender) that it can constrict wholesale.

In essence, the FCC's rule requires that ISPs behave ``neutrally''
with regards to traffic coming from content providers who
\textbf{aren't} their customers (or peers).

But what of the ISP's actual paying customers? By contrast, they
receive no guarantee of neutrality under the rule. Major ISPs freely
restrict their customers from sending packets on behalf of other
paying entities\footnote{Comcast and Google Fiber prohibit residential
  customers from reselling Internet service; Time Warner Cable does
  not.} or block certain Internet traffic to and from
customers, such as TCP port 25.

In this position paper, we argue that network neutrality got it
backwards. Internet growth is to be had by \emph{encouraging} paid
prioritization, while requiring ISPs to deal neutrally with
everybody. Just as an ISP can no longer discriminate among packets
sent by a noncustomer (e.g., RCN cannot prevent Amazon from reselling
its bandwidth to EC2 customers), we propose that an ISP should be
required to allow its own customers to resell their spare capacity as
well.

\subsection{Stickiness in today's Internet}

Today, Internet service is sticky---a typical consumer is loyal to one
ISP, who promises global connectivity for one bill. Transaction costs
to enter the market are considerable: an ISP must negotiate global
connectivity (via transit or peering agreements), then work to lure
customers away from their current providers.

Although Internet access is essentially a commodity, there is
substantial evidence that consumers are swayed by noneconomic
factors. For example, in 2014, Comcast spent \$3.1 billion on advertising and
marketing for its Cable Communications business,\footnote{This figure
  includes ads that promote both Internet and cable-TV connectivity.}
compared with \$2.0 billion in capital expenditures to improve its
cable distribution system~\cite{comcastannualreport}. By contrast,
mature commodities markets with sophisticated buyers (e.g.~coal,
natural gas) see little advertising.

At the same time, the Internet falls down at its ability to meet user
needs. Two decades ago, the IntServ RFC authors wrote that ``real-time
applications often do not work well across the Internet because of
variable queueing delays and congestion losses.''~\cite{rfc1633}
Despite meteoric growth of the Internet since then, the statement is
still true. There's no way for an ordinary Skype user to pay more to receive
guaranteed throughput and bounded latency, or for an e-mail sender to
pay less because of flexibility about when a message is sent. There's no
way to pay for interdomain multicast service.

In a world where an ISP provides branded global connectivity all by
itself, exciting models of communications networks remain
tantalizingly unrealized.  Google's attempt at a Wi-Fi mesh network
for Mountain View, Calif., ultimately became overloaded with traffic
and was shut down in 2014~\cite{pcworld13}. Cisco Meraki's mesh
network in San Francisco suffered a similar fate~\cite{economist14}.
The question of this work is: Would these networks have shut down, and
would urban mesh networks become viable, if individual mesh-node
operators were compensated for the value they were contributing
to the network's users?

\subsection{Towards self-incentivizing networks}

We believe that in designing new subnetwork technologies, the
networking-research community should consider how to make them
function as self-incentivizing enclaves of the Internet. In these
enclaves, transaction costs would be explicitly
considered and minimized.

As examples, we envision the following use-cases for a
self-incentivizing network that address needs unmet by existing systems:

\begin{itemize}
\item \textbf{Neighbor sharing:} Two neighbors with Internet access
  should be able to offer each other their spare capacity (and the
  chance of redundant service) for a price, settling up at the end of
  the month.

\item \textbf{Urban mesh network:} In a dense urban region, building
  owners should be able to install mesh nodes and be compensated for
  the value they contribute to the users. Before investing in a mesh
  node (or a mesh node with uplink capacity to the real Internet),
  they should be able to predict how much revenue they will earn and
  how long it will take them to recoup their investment, by looking at
  what users are currently offering to pay.

\item \textbf{Train:} Cellular Internet service on a commuter train
  often has dead spots. Passengers should not have to pay for capacity
  they're not getting, but network operators should be able to
  predict when it becomes worthwhile to invest in building a new
  cell-phone tower to fill in a dead spot. Passengers should be
  agnostic as to which provider does this.

\item \textbf{Wide-area service:} There should be a liquid market in
  ``whose packet goes next'' over high-capacity
  transcontinental and transoceanic links. Anybody should be able to
  add more capacity to a pair of endpoints and start bidding on that
  market, without having to persuade customers to abandon their
  loyalty to existing providers.

\end{itemize}

In these networks, anybody (even a retail ISP customer) would be able
to contribute incremental capacity and be compensated for it. In this
vision, applications and endpoints express how much they care about
getting any particular packet or group of packets through, in what
kind of timeframe, and network providers will compete to make that
happen.

We imagine that users would express their overall preferences similar
to today---perhaps by stating that they wish to pay about \$30/month
for Internet service, and leaving the actual bidding procedures up to
their computers. Instead of congestion-control protocols, we will have
``bidding-control protocols'' that achieve application-defined objectives in the
micro scale, and meet a user-specified budget in the macro scale.

Transport protocols would need to become roamable and robust to the
possibility that their upstream link could change at any moment
(necessitating a change in IP source address).\footnote{The recent
  emergence of single-packet roaming in QUIC, MOSH, and OpenVPN
  suggests this is feasible.}

We follow a long line of work on wide-area network economics and
bandwidth auctions, which we discuss further in
Section~\ref{sec:related}. But our focus is slightly different: for now, we concern ourselves
with the ability of a market mechanism to replace TCP and active-queue
management on a hop or small number of hops, without (yet) tackling
issues of interdomain routing.

In our simulation experiments so far, we demonstrate that \emph{on a
  single contended hop}, a market mechanism can recover very close to the
shortest-remaining-time-first schedule among distributed endpoints who
only interact with each other through a ``bidding-control protocol.''
This produces faster flow-completion times than an idealized TCP.

\subsection{Contributions}

The contributions of this position paper are:

\begin{itemize}

\item An architectural sketch suggesting that
  self-incentivizing networks could solve some currently-vexing problems in
  networking, without requiring a refactor of the whole Internet (this section).

\item A design of a simple market that replaces TCP and AQM over a
  single hop, which we evaluated in simulation. It has good
  performance where all users behave ``nicely,'' but is susceptible to
  mischief (Section~\ref{ss:simplemarket}).

\item The design of a multiresolution market that presents a range of
  compromises between throughput, jitter, and price and which we
  believe will be more robust (Section~\ref{ss:multires}).

\end{itemize}

Before discussing the contributions, we survey the related work in
Section~\ref{sec:related}. We close with a discussion of limitations
and future work (Section~\ref{sec:limits}).
