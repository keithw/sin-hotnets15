\section{Introduction}

\label{sec:intro}

On June 12, 2015, the Federal Communications Commission's Open
Internet rules went into effect. In pursuit of what's commonly known
as ``net neutrality,'' the agency bans ``retail'' Internet service
providers from ``blocking,'' ``throttling,'' or ``paid
prioritization'' of traffic across their networks~\cite{openinternet}.

The ``paid priorization'' rule is intended to protect third-party
content providers who aren't transit customers or peers of the
ISP. The rule prohibits retail ISPs from demanding money from a third
party or affiliate in exchange for ``traffic shaping, prioritization,
resource reservation, or other forms of preferential traffic
management'' of that party's traffic within the ISPs network.

An unaffiliated third party can send its own traffic, or somebody
else's traffic---the ISP cannot discriminate except to the extent that
it has a link at its border (a direct peering or transit connection to
the sender) that it can constrict wholesale.

In essence, the FCC's rule requires that ISPs behave ``neutrally''
with regards to traffic coming from content providers who
\textbf{aren't} their customers (or peers).

But what of the ISP's actual paying customers? By contrast, they
receive no guarantee of neutrality under the rule. Major ISPs freely
restrict users from sending packets on behalf of other paying
entities\footnote{Comcast and Google Fiber prohibit residential
  customers from reselling Internet service; Time Warner Cable does
  not.} or block certain Internet traffic to and from
customers.\footnote{E.g., TCP port 25.}

In this position paper, we argue that network neutrality got it backwards.



``real-time applications often do not work well across the Internet
because of variable queueing delays and congestion losses.''\cite{rfc1633}

\subsection{Policy considerations}
